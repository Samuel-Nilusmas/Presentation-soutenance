%% PRÉAMBULE : PACKAGES début


\usepackage[absolute,overlay]{textpos}
\usepackage{graphicx}
\setbeamertemplate{section in toc shaded}[default][65]
\setbeamertemplate{frametitle}[default][center]
\useoutertheme[section=false,subsection=false]{miniframes}
%			  \usepackage{lmodern}
%\renewcommand*\familydefault{\sfdefault} 
\usepackage{adjustbox}
\usepackage{blindtext}
\setbeamertemplate{section in toc}[sections numbered]
\defbeamertemplate{subection in toc}{subsections numbered roman}{%
\@Alph\inserttocsectionnumber.\ %
\inserttocsection\par}
\usepackage[french]{babel}
\usepackage[utf8]{inputenc}
\usepackage[T1]{fontenc}
\usepackage{hyperref}
\languagepath{French}
 \usepackage[cm]{sfmath}
\usepackage{here}
\usepackage{amsrefs} 
\usepackage{graphicx}
\usepackage{eurosym}
\usepackage{wasysym}
\usepackage{outlines}
\RequirePackage{tikz}
\usepackage{tikz}
\usepackage{times,amsmath,amssymb,pifont,array}
\usepackage{enumitem}
\usetikzlibrary{decorations.pathreplacing} 
\usepackage{booktabs}
\usepackage{array}
\usepackage{siunitx}


%\tikzposterlatexaffectionproofonfalse%shows small comment on how the poster was made at bottom of poster
\usepackage{ifxetex}
\usepackage{ifthen}

\ifxetex
    \usepackage{libertine}
  %  \usepackage[log-declarations=false]{xparse} % same as below
%  \usepackage[quiet]{fontspec} % remove quiet option to check for error
  \setromanfont[Ligatures={Common,TeX}]{Linux  Libertine O}
  \setmainfont[Ligatures={Common,TeX}]{Linux  Libertine O}
    \setmonofont[SmallCapsFont={Latin Modern Mono Caps}]{Latin Modern Mono Light}
    \setsansfont{Linux Biolinum O}
  
    \usepackage{xunicode}
    
%    \usepackage[lf]{venturis} %% lf option gives lining figures as default; 
%			  %% remove option to get oldstyle figures as default
%\renewcommand*\familydefault{\sfdefault} %% Only if the base font of the document is to be sans serif

\else
    \usepackage{sansmath}
%    \usepackage{lmodern}
    \usepackage{libertine}

 % \usepackage[utf8]{inputenc}
 % \usepackage[T1]{fontenc}
  \usepackage{textcomp}
  \usepackage{epstopdf}
     \usepackage[cm]{sfmath}
    \usepackage[lf]{venturis} %% lf option gives lining figures as default; 
			  %% remove option to get oldstyle figures as default
	\renewcommand*\familydefault{\sfdefault} %% Only if the base font of the document is to be sans serif		  


%% Only if the base font of the document is to be sans serif
  \usepackage[final,expansion=true,protrusion=true,spacing=true,kerning=true]{microtype}
\fi
%Theme
\usetheme{SN}

% Tableau
\usepackage{booktabs,makecell,multirow,tabularx}
\renewcommand\theadfont{\normalfont\bfseries}
\usepackage{multicol}


% Sous sous sous section
%\setbeamercolor{subsubsection in toc}{bg=white,fg=structure}
%\setbeamertemplate{subsubsection in toc}{%
 % \hspace{3em}{\rule[0.3ex]{3pt}{3pt}}~\inserttocsubsubsection\par}
%\documentclass[xcolor={dvipsnames},handout]{beamer}
%\usepackage{pgfpages}
%\pgfpagesuselayout{4 on 1}[a4paper,border shrink=5mm,landscape]


%% compatbilité beamer-enumitem
\setitemize{label=\usebeamerfont*{itemize item}%
  \usebeamercolor[fg]{itemize item}
  \usebeamertemplate{itemize item}}
\setenumerate[1]{%
  label=\protect\usebeamerfont{enumerate item}%
  \protect\usebeamercolor[fg]{enumerate item}%
  \insertenumlabel.}
\newcolumntype{C}[1]{>{\centering\arraybackslash }m{#1}}
\renewcommand\shapedefault{\scdefault\itdefault}
\usepackage[nomarkers,nolists,fighead,tablesfirst,notablist]{endfloat}


%% FLOATS = FIGURES & TABLES
\graphicspath{{./Figures/}}
\DeclareGraphicsExtensions{.pdf,.png,.jpg}
\AtBeginDelayedFloats{\renewcommand{\baselinestretch}{1.5}}
\renewcommand{\efloatseparator}{\mbox{}}

%Couleur
\definecolor{myorange}{RGB}{255, 122, 34}
\definecolor{myblue1}{RGB}{0,153,255}
\definecolor{myblue2}{RGB}{3, 101, 192}
\definecolor{myblue3}{RGB}{0,0,170}
\definecolor{myblue4}{RGB}{74,198,183}
\definecolor{mypink}{RGB}{242, 5, 159}
\definecolor{mygreen}{RGB}{1, 113, 0}
\definecolor{mypink2}{RGB}{255, 112, 112}
\definecolor{mypurple2}{RGB}{200, 25, 112}
\definecolor{mypurple}{RGB}{77, 5, 159}
\definecolor{mypink1}{rgb}{1.0, 0.01, 0.24}
\definecolor{myyellow}{RGB}{255, 165, 0}
\definecolor{myvert}{RGB}{60, 138, 158}
\definecolor{myor}{RGB}{191, 116, 65}
\definecolor{myblue}{rgb}{0.33, 0.37, 0.51}
\definecolor{shadecolor}{rgb}{0.01,0.199,0.1}
\colorlet{shadecolor}{gray!40}
\colorlet{shadecolor}{gray!90}
\setbeamertemplate{footline}[frame number]
\definecolor{cor3}{RGB}{30,130,186}
\colorlet{couleurs}{cor3}
\colorlet{couleur}{myvert}
\definecolor{vertmoyen}{rgb}{0.20,0.43,0.09}
\setbeamercolor{structure}{fg=couleur}
\setbeamercolor{alerted text}{fg=couleur} 
\setbeamercolor{darkbox}{fg=black,bg=white!75!blue}
\setbeamercolor{lightbox}{fg=black,bg=white!90!blue}
\setbeamertemplate{blocks}[rounded]
\setbeamercolor{block title}{fg=couleur,bg=white!90!blue}
\setbeamercolor{block body}{fg=black,bg=white!90!blue}
\setbeamercolor{block title alerted}{fg=couleur,bg=white!90!blue}
\setbeamercolor{block body alerted}{fg=black,bg=white!90!blue}
\newcommand{\Sb}{\includegraphics[width=4mm]{S.pdf}}
\newcommand{\Rb}{\includegraphics[width=4mm]{R.pdf}}
\definecolor{vertclair}{RGB}{100,136,150} 
\beamertemplateshadingbackground{blue!10!lightgray!10!white} {blue!10!white} 



%% FORMAT PAGE
%Marges
\setbeamersize{text margin left=15pt,text margin right=15pt}
\setlength\parindent{0pt}
\setlength{\unitlength}{1cm} 



%% PRÉAMBULE : STYLE 
\setbeamercovered{invisible}




%% Ombre image
\usetikzlibrary{shadows,calc}
 
\def\shadowshift{3pt,-3pt}
\def\shadowradius{5pt}
 
\colorlet{innercolor}{black!60}
\colorlet{outercolor}{gray!05}
 
\newcommand\drawshadow[1]{
    \begin{pgfonlayer}{shadow}
        % Effet d'ombre en forme de cercle en bas à gauche
        \shade[outercolor, inner color=innercolor, outer color=outercolor] ($(#1.south west)+(\shadowshift)+(\shadowradius/2 , \shadowradius/2)$) circle (\shadowradius);
        % Effet d'ombre en forme de cercle en bas à droite
        \shade[outercolor, inner color=innercolor, outer color=outercolor] ($(#1.south east)+(\shadowshift)+(-\shadowradius/2 , \shadowradius/2)$) circle (\shadowradius);
        % Effet d'ombre en forme de cercle en haut à droite
        \shade[outercolor,inner color=innercolor,outer color=outercolor] ($(#1.north east)+(\shadowshift)+(-\shadowradius/2 , -\shadowradius/2)$) circle (\shadowradius);
 
        % Dégradé de haut en bas sur la partie sud du rectangle
        \shade[top color=innercolor,bottom color=outercolor] ($(#1.south west)+(\shadowshift)+(\shadowradius/2,-\shadowradius/2)$) rectangle ($(#1.south east)+(\shadowshift)+(-\shadowradius/2,\shadowradius/2)$);
        % Dégradé de gauche à droite sur le côté droit du rectangle
        \shade[left color=innercolor,right color=outercolor] ($(#1.south east)+(\shadowshift)+(-\shadowradius/2,\shadowradius/2)$) rectangle ($(#1.north east)+(\shadowshift)+(\shadowradius/2,-\shadowradius/2)$);
 
        % On remplit le rectangle créé avec une couleur noire
        \filldraw ($(#1.south west)+(\shadowshift)+(\shadowradius/2,\shadowradius/2)$) rectangle ($(#1.north east)+(\shadowshift)-(\shadowradius/2,\shadowradius/2)$);
    \end{pgfonlayer}
}
 
\pgfdeclarelayer{shadow}
\pgfsetlayers{shadow,main}
 
\newcommand\shadowimage[2][]{
    \begin{tikzpicture}
        \node[anchor=south west,inner sep=0] (image) at (0,0) {\includegraphics[#1]{#2}};
        \drawshadow{image}
    \end{tikzpicture}
}



